\section{Experimental Results}
\label{sec:experiments}

\subsection{Full Few-Shot Learning}
To answer question (1), we first explore the few-shot learning scenario.

\paragraph{Intent Detection}

\begin{table*}
\setlength{\tabcolsep}{0.23em}
\centering
\small{
\begin{tabular}{|c|c|c|c|c|c|c|c|}

\hline
\multirow{3}{*}{\textbf{Model Type}} & \multirow{3}{*}{\textbf{Model Name}}  & \multicolumn{3}{|c|}{\textbf{Intent}} & \multicolumn{3}{|c|}{\textbf{Slot}} \\
\cline{3-8}
&  & \multicolumn{1}{|c|}{\textbf{5-shot}} & \multicolumn{1}{|c|}{\textbf{10-shot}} & \multicolumn{1}{|c|}{\textbf{20-shot}} 
& \multicolumn{1}{|c|}{\textbf{5-shot}} & \multicolumn{1}{|c|}{\textbf{10-shot}} & \multicolumn{1}{|c|}{\textbf{20-shot}}  \\
\cline{3-8}
&  & \multicolumn{3}{|c|}{\textbf{Macro-F1 / Accuracy}} & \multicolumn{3}{|c|}{\textbf{Macro-F1 / Accuracy}} \\
\hline
Base Model & BLSTM & 45.28 / 60.02 & 60.62 / 64.61 & 63.60 / 80.52 
& 60.78 / 83.91 & 74.28  / 90.19 & 80.57 / 93.08  \\
\hline
Input Side & +feat & 49.40 / 63.72 & 64.34 / 73.46 & 65.16 / 83.20 
& \textbf{66.84} / \textbf{88.96} & 79.67 / \textbf{93.64} & 84.95 / 95.00  \\
\hline
\multirow{2}{*}{Output Side} & +logit & 46.01 / 58.68 & 63.51 / 77.83 & 69.22 / \textbf{89.25} 
& 63.68 / 86.18 & 76.12 / 91.64  & 83.71 / 94.43 \\
\cline{2-8}
& +hu16 & 47.22 / 56.22 & 61.83 / 68.42 & 67.40 / 84.10 
& 63.37 / 85.37 & 75.67 / 91.06 & 80.85 / 93.47  \\
\hline
\multirow{2}{*}{\vspace{-2.2em}NN Module Side} & +t\_att & 40.44 / 57.22 & 60.72 / 75.14 & 62.88 / 83.65
& 60.38 / 83.63 & 73.22 / 90.08 & 79.58 / 92.57  \\
\cline{2-8}
& +t\_att+posi & 50.90 / 74.47 & 68.69 / 84.66 & 72.43 / 85.78 
& 59.59 / 83.47 & 73.62 / 89.28 & 78.94 / 92.21 \\
\cline{2-8}
& +t\_att+neg & 49.01 / 68.31 & 64.67 / 79.17 & 72.32 / 86.34 
& 59.51 / 83.23 & 72.92 / 89.11 & 78.83 / 92.07 \\
\cline{2-8}
& +t\_att+both & \textbf{54.86} / \textbf{75.36} & \textbf{71.23} / \textbf{85.44} & \textbf{75.58} / 88.80 
& 59.47 / 83.35 & 73.55 / 89.54 & 79.02 / 92.22 \\
\hline
\multirow{2}{*}{Few-Shot Model} & +mem & - & - & - & 61.25 / 83.45 & 77.83 / 90.57 & 82.98 / 93.49 \\
\cline{2-8}
& +mem+feat & - & - & - & 65.08 / 88.07 & \textbf{80.64} / 93.47 & \textbf{85.45} / \textbf{95.39} \\
\hline
\hline
RE Output & RE & \multicolumn{3}{|c|}{70.31 / 68.98} & \multicolumn{3}{|c|}{42.33 / 70.79} \\
\hline
\end{tabular}
}
\caption{Results on Full Few-Shot Learning Setting}
\label{tab_full_few}
\end{table*}

As shown in Table \ref{tab_full_few}, except for 5-shot, all the methods improve the baseline \texttt{BLSTM}. 
Among these methods, our \NN module side methods (attention-loss-based methods) work best.
The reason is that, the attention module receives the clue words of this prediction directly, which contain more information than the \RE tag used by other methods.
We can also see that, since negative patterns are derived from the positive ones with some noise, \texttt{posi} performs better than \texttt{neg} when the data is limited.
However, \texttt{neg} works slightly better in the 20-shot setting, which is possibly due to that negative patterns significantly outnumbers the positive ones. 
% Not surprisingly, we get the best results when positive and negative patterns are combined. 
Besides, \tatt alone works better than \texttt{BLSTM} when we have enough data, indicating again the effectiveness of our two-side attention method.

Further, we can also see that the output side method (\texttt{logit}) works generally better than input side one (\texttt{feat}), except for the 5-shot case.
The reason is probably that, the less \RE related parameters and the shorter distance from the modification to the final output makes \texttt{logit} easier to learn.
However, since \texttt{logit} modifies output directly, the final prediction is more sensitive to the insufficiently trained parameters of \texttt{logit}, which leads to its worse results in the 5-shot setting. 

To compare with existing methods of combining \NN and rules, we also implement the teacher-student network \cite{liu2016attention}, which proves to perform well on various datasets. 
They first use \FOL rules to rectify the label distribution produced by \NN, and then let the \NN to learn from the rectified distribution, which requires a certain amount of data to support this process.
Therefore, although \texttt{hu16} and \texttt{logit} are both output side methods, since \texttt{logit} is easier to learn,
\texttt{hu16} performs consistently worse than \texttt{logit} in this few-shot learning setting.
% \todo{try to acknowledge that \texttt{hu16} belongs to \emph{fusion in output}}

We can also see that, the improvements from \RE decreases as the size of training data increases, 
which is reasonable since as we have more data, the information contained in the data will have more overlap with the information in \RE. 

Besides, starting from 10-shot, \texttt{t\_att+both} significantly outperforms pure \RE.
This shows that, by using our attention loss to connect the distrional representation of \NN and the clue words of \RE, we can generalize \RE patterns by using only a small amount of data.


\paragraph{Slot Filling}

Different from intent detection, as shown in Table \ref{tab_full_few}, the attention loss does not work for slot filling.
% \footnote{Negative pattern works even worse, we therefore only show the results of positive pattern here}. 
The reason is that, the slot label of a \textbf{\emph{target word}} (the word that we are trying to predict slot label) is decided mainly by the word itself, together with 0-3 words in the context to provide supplementary information.
Therefore, since attenion can only help recognizing clue words in the context, which can be captured by \BLSTM to some extent and is less important than the word itself, the attention is not that useful as in intent detection.
Therefore, extra parameters introduced by attention and the attention loss are more of a burden than benifit.
% However, attention does not contribute to better self-awareness. 
% Since \BLSTM output already models some context, and attention does not help understanding the word itself, the attention is not that useful as in intent detection, making the attention loss more of a burden than benifit.
Take the sentence in Table \ref{atis_sample} again for example, we recognize \textsl{\underline{boston}} as \emph{fromloc.city} because the word \textsl{\underline{boston}} reprensents a city, and it follows a context word \emph{from}. Since this simple context is easily captured by \BLSTM, attention does not help much in this case.
By examing the attention values of \texttt{t\_att} trained on full dataset,
we find instead of marking informative context words, the attention tends to concentrate on the word that is to be predicted itself, which further confirms our hypothesis on attention loss.

However, due to the word list \RE groups like city list, the \RE tag actually provides type information of the target word. 
Therefore \texttt{logit} and \texttt{feat} works better here.
However, different from intent detection, \texttt{feat} performs better this time.
The reason is that, since \BLSTM need the type of the target word to better model the context, \texttt{feat} actually makes better use of the information from \RE than \texttt{logit}.
Further, by converting \RE tag to slot label, \texttt{logit} also introduces extra noise to the model.
As for \texttt{hu16}, with the same reason mentioned in the intent part, it still performs consistently worse than \texttt{logit}.

% It is also interesting that, \texttt{posi} and \texttt{feat} is complementary to each other. \texttt{posi+feat} performs better than pure \texttt{feat} itselft when we have enough data. This is probably because the extra light-weight NER from \texttt{feat} significantly reduces the need for self-awareness of the target word, which leaves possibility to better model the context (using attention).

We can see that even \texttt{BLSTM} outperforms \RE (see the \RE row in Table \ref{tab_full}) in 5-shot, showing that it is hard to write high-performance \RE patterns, but using \RE to boost \NN is feasible. 
% It is not surprising because this is not true 5-shot setting, extra data still exists for frequent patterns since one sentence may contain multiple slots.


\subsection{Partial Few-Shot Learning}
To better understand the relationship between our method and existing few-shot learning methods, we also implement the memory network method \cite{kaiser2017learning}, which achieves good results in various few-shot settings. Specifically, by adapting their open-source code, we add their memory module (\texttt{mem}) to our \BLSTM model.

\begin{table}
\setlength{\tabcolsep}{0.23em}
\centering
\small{
\begin{tabular}{|c|c|c|c|}

\hline
\multirow{2}{*}{\textbf{Model}}  & \multicolumn{1}{|c|}{\textbf{5-shot}} & \multicolumn{1}{|c|}{\textbf{10-shot}} & \multicolumn{1}{|c|}{\textbf{20-shot}}  \\
\cline{2-4}
 & \multicolumn{3}{|c|}{\textbf{Macro-F1 / Accuracy}}   \\
\hline
BLSTM & 64.73 / 91.71 & 78.55 / 96.53 & 82.05 / 97.20 \\
\hline
+hu16 & 65.22 / 91.94 & 84.49 / 96.75 & 84.80 / 97.42 \\
\hline
+t\_att & 65.59 / 91.04 & 77.92 / 95.52 & 81.01 / 96.86 \\
\hline
+t\_att+both & 66.62 / 92.05 & 85.75 / 96.98 & \textbf{87.97} / \textbf{97.76} \\
\hline
+mem & 67.54 / 91.83 & 82.16 / 96.75 & 84.69 / 97.42 \\
\hline
+mem+posi & \textbf{70.46} / \textbf{93.06} & \textbf{86.03} / \textbf{97.09} & 86.69 / 97.65 \\
\hline

\end{tabular}
}
\caption{Intent Detection Results on Partial Few-Shot Learning Setting.}
\label{tab_intent_few_fill}
\end{table}

Since the memory module requires to be trained on either many few-shot classes or several classes with extra data.
Therefore, we expand our full few-shot dataset for intent detection, so that the top 3 intents have 300 sentences (partial few-shot).

As shown in Table~\ref{tab_intent_few_fill}, while \texttt{mem} works better than the base model, our attention loss also makes clear improvements in this setting.
\texttt{hu16} also works here, but is inferior to \texttt{t\_att+both}.
Further, the attention loss can also be combined with the memory module (\texttt{mem+posi}), and achieves better results than \texttt{mem} alone. 
Note that, since the attention module requires the input sentence to have only one embedding, we only use one set of possitive attention in this case.

As for slot filling, since we already have extra data for frequent tags in the original few-shot data, we use them directly to run the memory module. As shown in Table \ref{tab_full_few}, \texttt{mem} also improves the base model, and it is also compatible to our \texttt{feat} fusion method, which gives \texttt{mem} a further boost.

For compactness, we only combine the best-performing fusion method in each task with \texttt{mem}. Other methods can be easily combined as well\footnote{
\texttt{logit} may not be able to applied to methods with other output form, e.g., matching based method~\cite{koch2015siamese}.}.

% NOTE: since we have more data here, the importance of attention guidance is not that crucial as in the previous more strict few-shot leanring setting. Therefore, the weight of attention loss is reduced to 1 in this setting and the following full dataset setting, 


\subsection{Full Dataset}

\begin{table}
\setlength{\tabcolsep}{0.23em}
\centering
\small{
\begin{tabular}{|c|c|c|}

\hline
\multirow{2}{*}{\textbf{Model}} & \textbf{Intent} & \textbf{Slot} \\ 
\cline{2-3}
  & \textbf{Macro-F1/Accuracy} &  \textbf{Macro-F1/Micro-F1} \\
\hline
BLSTM & 92.50 / 98.77  & 85.01 / 95.47\\
\hline
+feat & 91.86 / 97.65 & 86.7 / 95.55\\
\hline
+logit & 92.48 / 98.77 & 86.94 / 95.42  \\
\hline
+hu16 & 93.09 / 98.77 & 85.74 / 95.33  \\
\hline
+t\_att & 93.64 / 98.88  & 84.45 / 95.05\\
% \hline
% two+posi & 93.02 & 98.88 \\
% \hline
% two+neg & 93.92 & \textbf{98.99} \\
\hline
+t\_att+both & \textbf{96.20} / \textbf{98.99} & 85.44 / 95.27 \\
\hline
+mem & 93.42 / 98.77 & 85.72 / 95.37\\
\hline
+mem+posi/feat & 94.36 / \textbf{98.99} & \textbf{87.82} / \textbf{95.90} \\
\hline
\hline
L\&L16 & - / 98.43 & - / 95.98\\
\hline 

\end{tabular}
}
\caption{Results on Full Dataset. The left side of \texttt{mem+posi/feat} applies for intent, and the right side applies for slot.} 
\label{tab_full}
\end{table}

To answer question (2), we also evaluate our methods on the full dataset. 
As seen in Table \ref{tab_full}, for intent detection,
the clue words marked by \RE still provide extra information to \NN (\texttt{t\_att+both}),
but the \RE tag is not very useful (\texttt{feat} and \texttt{logit}).
The reason is probably that, comparated with the knowledge learned from data, the \RE tag is much more noisy.
Further, since \texttt{feat} operates on the input side, the powerful \NN makes it more likely to overfit than \texttt{logit}, and therefore performs even inferior the baseline \BLSTM.
% This shows that, although the informative words marked by \RE still helps, the low-performance \RE output no longer works when the labeled data is sufficient. While \NN can learn to assign low weights for \RE output in \texttt{logit}, the noise coming from the input feature is hard to eliminate, and therefore leading to bad results of \texttt{feat}.

As for slot filling, however, \texttt{feat} and \texttt{logit} still works. 
The reason is that, the word type information contained in the \RE tag is still hard to be learned even when we have more data. 
% the output of the \RE used for \texttt{feat} and \texttt{logit} provides extra information to help \NN better understand the target word (e.g., entity type), which is hard to inferred by \NN itself.
% The reason is probably that the output of the \RE used for \texttt{feat} and \texttt{logit} is only the entity part of the slot label, which is only indirectly connected to the prediction target. The noise contained in the indirectness makes \NN relies less on the \RE output, and therefore is less sensitive to the wrong predictions made by \RE as they do in intent detection.
Also note that, different from few-shot settings, \texttt{t\_att+both} has better macro-F1 than \texttt{BLSTM} here, showing that better attention is still useful when base model is properly trained. 

In intent detection, \texttt{hu16} performs better than \texttt{logit}, showing that \NN can still learn some useful information from the noisy intent \RE tags when we have enough data.
However, since \texttt{hu16} is a general framework to combine \FOL rules, it is more indirect in transfering knowledge from rules to \NN than our methods. Therefore, its performance is still clearly inferior to attention loss in intent detection and \texttt{feat} in slot filling, which are designed specifically for \RE rules.
% The reason that it still performs worse than attention loss intent detection and in slot filling, is 

Further, \texttt{mem} still generally works, and we can also make improvements by further combining our fusion methods. 
Besides, we can also see that our base model achieves comparative results to the joint model of  Liu and Lane~\shortcite{liu2016attention}, which achieves state-of-art results on the ATIS data\footnote{
Since slot filling is evaluated in phrase level, there is almost no difference in F1 when we convert the prediction on our split data format to the original format (see Section \ref{sec_datasest})}.
% Besides, the state-of-art results on the ATIS data produced by \cite{liu2016attention} is also included (\LL). We can see that our base \BLSTM model achieves comparative results to \LL, confirming that the improvements from our fusion method does not come from the inferior ability of the base model), making our results still comparatable to \LL.}. 
And the analysis on different settings in the sections above together answer question (3).

\subsection{Complexity of the Pattern}
\label{sec_complexity}
\begin{table}
\setlength{\tabcolsep}{0.23em}
\centering
\small{
\begin{tabular}{|c|c|c|c|c|}

\hline
\multirow{3}{*}{\textbf{Model}}  & \multicolumn{2}{|c|}{\textbf{Intent}} & \multicolumn{2}{|c|}{\textbf{Slot}}  \\
\cline{2-5}
  & \multicolumn{2}{|c|}{\textbf{Macro-F1 / Accuracy}} & \multicolumn{2}{|c|}{\textbf{Macro-F1 / Micro-F1}}  \\
\cline{2-5}
  & \textbf{Before} & \textbf{After} & \textbf{Before} & \textbf{After} \\
\hline
BLSTM & \multicolumn{2}{|c|}{63.60 / 80.52} & \multicolumn{2}{|c|}{80.57 / 93.08}  \\
\hline
+feat & 65.16/\textbf{83.20} & \textbf{66.51}/80.40 & \textbf{84.95/95.00} & 83.88/94.71 \\
\hline
+logit & \textbf{69.22/89.25} & 65.09/83.09 & \textbf{83.71/94.43} & 83.22/93.94  \\
\hline
+both & \textbf{75.58/88.80} & 74.51/87.46 & - & - \\
\hline 
\end{tabular}
}
\caption{Results on 20-Shot Data with Simple Patterns. \texttt{+both} refers to \ptatt\texttt{+both} for short.}
\label{tab_simple}
\end{table}

This section tries to answer how the \RE complexity affects the fusion performance.
Since enriching the phrases in an \RE group with $|$ is usually easier adding new groups,
we choose to control the \RE complexity by modifying the number of groups.
As discussed in Sec.~\ref{re_desc}, more \RE groups will lead to better precision but lower coverage.

Specifically, we reduce the number of groups of existing \texttt{RE}s to decrease pattern complexity.
To mimic the process of writing simple \texttt{RE}s from scratch, we try our best to keep the key \RE groups.
For intent detection, all the \texttt{RE}s are reduced to at most 2 groups (groups matching a sequence of any words are excluded), and 17 \texttt{RE}s are affected. 
As for slot filling, since the \texttt{RE}s with more than 2 groups are limited (only 7), we choose a more aggresive strategy that all the \texttt{RE}s are reduced to word list pattern.
However, the \texttt{RE}s trying to tag an \RE group that matches a number or a common word can have at most 2 groups, since it is not distinguishable at all without some context.
In total, 21 patterns are affected.

As shown in Table \ref{tab_simple}, simple \RE generally leads to worse results, which indicates that we should write complex \RE to get better performance if the cost is affordable.
Further, the results of simplified \RE are still better than the base \BLSTM, showing that in practice, we can safely start with simple \texttt{RE}s, and increase the complexity gradually\footnote{
% Further, we can see that attention loss is less sensitive to the simplification than other methods. 
% The reason probably lies in that, although simplified, the core informative words still remains in the pattern, which is therefore still helpful for the attention module
We do not include simplification results of attention loss for slot filling because its \texttt{RE}s are different from the other two methods, and we already know it does not work for slot filling}.

\subsection{Key Conclusion}
We summarize the key conclusions from the experiments above here: 
(1) The amount of extra information that \NN can learn from \RE influences the fusion performance most. Therefore, attention loss methods works best for intent detection and \texttt{feat} works best for slot filling.
(2) The improvements from \RE decreases as we have more training data.
(3) Our fusion methods are compatible with existing few-shot learning methods like \texttt{mem}.
(4) Complex \RE leads to better fusion performance.
\todo{Is this section necessary?}

\section{Evaluation Methodology}
Our experiments
aim to %are designed
answer three questions: \textbf{Q1:} Does the use of \REs enhance the learning quality when the number of annotated
instances is small?  \textbf{Q2:}  Does the use of \REs still help when using the full training data?
 \textbf{Q3:}  How can we choose from different combination methods?
%When do the three proposed methods work, and which method works best in each scenario?
% \todo{how to refer to question (3) in exp?}

\subsection{Datasets}
\label{sec_datasest}

We use the ATIS dataset~\cite{hemphill1990atis} to evaluate our approach. This dataset is widely used in \SLU research. It includes queries
of flights, meal, etc. We follow the setup of Liu and Lane~\shortcite{liu2016attention} by using 4,978 queries for training and 893 for
testing, with 18 intent labels and 127 slot labels.
% Numbers are replaced with special tokens like \textsl{\underline{DIGIT*m}},
% where \emph{m} is the number of digits in the original string.
%Different from previous work,
We also split words like \textsl{\underline{Miami's}} into \textsl{\underline{Miami 's}} during the tokenization phase to reduce the number
of words that do not have a pre-trained word embedding. This strategy is useful for few-shot learning.

To answer  \textbf{Q1} , we also exploit the \textbf{\emph{full few-shot learning setting}}. Specifically, for intent detection, we
randomly select 5, 10, 20 training instances for each intent to form the few-shot training set; and for slot filling, we also explore 5,
10, 20 shots settings. However, since a sentence typically contains multiple slots, the number of mentions of frequent slot labels may
inevitably exceeds the target shot count. To better approximate the target shot count, we select sentences for each slot label  in
ascending order of label frequencies. That is $k_1$-shot dataset will contain $k_2$-shot dataset if $k_1>k_2$. All settings use the
original test set.

Since most existing few-shot learning methods require either many few-shot classes or some classes with enough data for training, we also
explore the \textbf{\emph{partial few-shot learning setting}} for intent detection to provide a fair comparison for existing few-shot
learning methods. Specifically, we let the 3 most frequent intents have 300 training instances, and the rest remains untouched. This is
also a common scenario in real world, where we often have several frequent classes and many classes with limited data. As for slot filling,
however, since the number of mentions of frequent slot labels already exceeds the target shot count, the original slot filling few-shot
dataset can be directly used to train existing few-shot learning methods. Therefore, we do not distinguish full and partial few-shot
learning for slot filling.

\subsection{Preparing REs}
\label{re_in_exp} We use the syntax of \REs in Perl in this work. Our \REs are written by a paid annotator who is familiar with the domain.
% \footnote{Code and data are available at: [url redacted for double-blind review].}.
It took the annotator in total less than 10 hours
to develop all the \REs, while a domain expert can accomplish the task faster.
We use the 20-shot training data to develop the \REs, but word lists
like cities are obtained from the full training set.
The development of \REs is considered completed when the \REs can cover most of the cases in the 20-shot training data with resonable precision.
After that, the \REs are fixed throughout the experiments.

The majority of the time
for %spent on
writing the \REs is proportional to the number of \RE groups.
It took about 1.5 hours to write the 54 intent \REs with on average 2.2 groups per \RE. It is straightforward
to write the slot \REs for the input and output level methods, for which it took around
 1 hour to write the 60 \REs with 1.7 groups on average. By contrast, writing slot \REs to guide attention requires more
efforts as the annotator needs to carefully select clue words and annotate the full slot label. As a result, it took about
5.5 hours to generate 115 \REs with on average 3.3 groups.
The performance of the \REs can be found in the last line of Table~\ref{tab_full_few}.

In practice, a positive \RE for intent (or slot) $k$ can often be treated as negative \REs for other intents (or slots). As such, we use the positive \REs for intent (or slot) $k$ as the negative \REs for other intents (or slots) in our experiments.


%Writing an intent \RE takes about 1-2 minutes. In this work, the student spent 1.5 hours to write the 54 \REs, with on average 2.2 \RE
%groups for each \RE. The student finds that writing the slot \RE for the methods described in Sec.~\ref{fusion_with_input} and
%\ref{fusion_with_output} to be easy, since we only annotate a simplified version of the slot label. On average, it takes less than two
%minutes to write a \RE. It took the student about an hour to write the 60 \REs with on average 1.7 \RE group. In contrast to the two
%aforementioned tasks, writing slot \REs to guide attention requires more efforts. This is because we need to carefully select clue words
%and target for the full slot label. As a result, it took between 2 to 5 minutes to write one \RE, yielding in total 5.5 hours for writing
%115 \REs with on average 3.3 \RE groups. The performance of the intent \REs and the slot \REs for the method described in
%Sec.~\ref{interact_with_module} is shown in Table~\ref{tab_full_few}.

%We use the standard \RE grammar in this work. Our \REs are written by an undergraduate who is familiar with the dataset\footnote{Code and
%data are available at: [url redacted for double-blind review].}.  It took the student in total less than 10 hours to develop all the \REs,
%but a domain expert can accomplish the task faster. We use the 20-shot data to develop the \REs, but word lists like cities are obtained
%from the full training set. The majority of the time spent on writing the \REs is proportional to the number of \RE groups. Writing an
%intent \RE takes about 1-2 minutes. In this work, the student spent 1.5 hours to write the 54 \REs, with on average 2.2 \RE groups for each
%\RE. The student finds that writing the slot \RE for the methods described in Sec.~\ref{fusion_with_input} and \ref{fusion_with_output} to
%be easy, since we only annotate a simplified version of the slot label. On average, it takes less than two minutes to write a \RE. It took
%the student about an hour to write the 60 \REs with on average 1.7 \RE group. In contrast to the two aforementioned tasks, writing slot
%\REs to guide attention requires more efforts. This is because we need to carefully select clue words and target for the full slot label.
%As a result, it took between 2 to 5 minutes to write one \RE, yielding in total 5.5 hours for writing 115 \REs with on average 3.3 \RE
%groups. The performance of the intent \REs and the slot \REs for the method described in Sec.~\ref{interact_with_module} is shown in
%Table~\ref{tab_full_few}.


% 54 patterns are collected in total
%\orange{(used 1.5 hours)}, with averagely 2.2 (from 1 to 4) \RE groups for each \RE (the group matching a sequence of any words is not
%included). Similarly, writing slot patterns for methods in Sec.~\ref{fusion_with_input} and \ref{fusion_with_output} is also easy since we
%only annotate a simplified version of the slot label. It takes 1-2 minutes to write an \RE, and 60 patterns are produced \orange{(used 70
%minutes)}, with averagely 1.7 (from 1 to 3) \RE groups. However, writing slot patterns to guide attention is more difficult, since we need
%to carefully select informative words and target to the full slot label as well. Typically, writing one pattern requires 2-5 minutes, and
%115 patterns \orange{(used 5.5 hours)} with averagely 3.3 (from 2 to 8) \RE groups are produced. The performance of the intent \texttt{RE}s
%and the slot \texttt{RE}s for the method in Sec.~\ref{interact_with_module} is shown in Table~\ref{tab_full_few}. \FIXME{ZW: We need to
%state in total how long did it take to derive all the REs. State which \RE convention is used. } \orange{LUO to ZW: see the newly added
%orange parts}


\subsection{Experimental Setup}
\cparagraph{Hyper-parameters} Our hyper-parameters for the \BLSTM are similar to the ones used by Liu and
Lane~\shortcite{liu2016attention}. Specifically, we use batch size 16, dropout probability 0.5, and \BLSTM cell size 100. The attention
loss weight is 16 (both positive and negative) for full few-shot learning settings and 1 for other settings. We use the 100d GloVe word
vectors ~\cite{pennington2014glove} pre-trained on Wikipedia and Gigaword~\cite{parker2011english}, and the Adam
optimizer~\cite{kingma2014adam} with learning rate 0.001.

\cparagraph{Evaluation Metrics}
We report accuracy and macro-F1 for intent detection, and micro/macro-F1 for slot filling.
Micro/macro-F1 are the harmonic mean of micro/macro precision and recall.
Macro-precision/recall are calculated by averaging precision/recall of each label, and micro-precision/recall are averaged over each prediction.
% While accuracy and micro statistics show performance on all the instances, macro statitics are more sensitive to classes with limited data.

\cparagraph{Competitors and Naming Conventions} Here, a bold Courier typeface like \textbf{\BLSTM} denotes the notations of the models that
we will compare in Sec.~\ref{sec:experiments}.

Specifically, we compare our methods with the baseline \textbf{\texttt{BLSTM}} model (Sec.~\ref{sec:baseline}). Since our attention loss
method (Sec.~\ref{interact_with_module}) uses two-side attention, we include the raw two-side attention model without attention loss
(\textbf{\ptatt}) for comparison as well. Besides, we also evaluate the \RE output (\textbf{\texttt{REO}}), which uses the \REtags as
prediction directly, to show the quality of the \REs that we will use in the experiments.\footnote{
	For slot filling, we evaluate the \REs that use the target slot labels as \REtags.}

As for our methods for combinging \REs with \NN,
\textbf{\texttt{+feat}}  refers to using \REtag as input features (Sec.~\ref{fusion_with_input}),
\textbf{\texttt{+posi}} and \textbf{\texttt{+neg}} refer to using positive and negative attention loss respectively,
\textbf{\texttt{+both}} refers to using both postive and negative attention losses (Sec.~\ref{interact_with_module}),
and \textbf{\texttt{+logit}} means using \REtag to modify \NN output (Sec.~\ref{fusion_with_output}).

Moverover, since the \REs can also be formatted as first-order-logic (\texttt{FOL}) rules, we also compare our methods with the
teacher-student framework proposed by Hu et al.~\shortcite{hu2016harnessing}, which is a general framework for distilling knowledge from
\texttt{FOL} rules into \NN (\textbf{\texttt{+hu16}}). Besides, since we consider few-short learning,  we also include the memory module
proposed by Kaiser et al.~\shortcite{kaiser2017learning}, which performs well in various few-shot datasets
(\textbf{\texttt{+mem}})\footnote{
	We tune $C$ and $\pi_0$ of \texttt{hu16}, and choose (0.1, 0.3) for intent, and (1, 0.3) for slot. We tune memory-size and $k$ of
\texttt{mem}, and choose (1024, 64) for intent, and (2048, 64) for slot. }. Finally, the state-of-art model on the ATIS dataset is also
included (\textbf{\LL}), which jointly models the intent detection and slot filling in a single network~\cite{liu2016attention}.


% \cparagraph{Naming Conventions}
% % We use the following naming conventions when discussing our experimental results:
% %
% \texttt{BLSTM} is our base model,
% \texttt{+feat} means using \REtag as input features (Sec.~\ref{fusion_with_input}),
% % input side method in
% \ptatt is the two-side attention method without attention loss,
% \texttt{+posi} and \texttt{+neg} refer to using positive and negative attention loss, respectively, \texttt{+both} refers to using both attention losses (Sec.~\ref{interact_with_module}),
% % \NN module side method in
% \texttt{+logit} means using \REtag to modify \NN output (Sec.~\ref{fusion_with_output}),
% % output side methods in
% \texttt{RE} refers to using \RE output as prediction directly\footnote{
% For slot filling, we evaluate the \REs that use the target slot labels as \REtags.},
% \texttt{+hu16} is the method of Hu et al.~\shortcite{hu2016harnessing},
% \texttt{+mem} is the memory module of Kaiser et al.~\shortcite{kaiser2017learning}\footnote{
% We tune $C$ and $\pi_0$ of \texttt{hu16}, and choose (0.1, 0.3) for intent, and (1, 0.3) for slot. We tune memory-size and $k$ of \texttt{mem}, and choose (1024, 64) for intent, and (2048, 64) for slot.
% },
% \LL is the state-of-art joint model~\cite{liu2016attention} in ATIS.

\section{Introduction}


%One of the fundamental techniques in computer science is the
The regular expression (\RE), a language for specifying text search strings, is widely used in various natural language processing (\NLP)
tasks like pattern matching, sentence classification, sequence labeling, etc.~\cite{chang2014tokensregex}. In dialogue systems, a \RE
pattern \texttt{/\textasciicircum flights? from/} can help recognize the sentence in Fig.~\ref{atis_sample} has an intention of
\emph{flights}, indicating the user is looking for flight-related information. As a technique based on human-generated rules, it is
concise, explainable, tunable, and does not rely on much training data, which is widely adopted in industry -- especially in scenarios
where the number of available training examples is small -- a problem known as few-shot learning~\cite{gc2015big}.

While powerful, \REs have a poor generalization ability because all synonyms and variations in a \RE must be explicitly specified. As a
result, they often have a low coverage of \z{??}. To overcome this limitation, \REs are often combined with data-driven methods like neural
network (\NN) based techniques. In this way, \REs are considered as an easy-to-tune method which, hopefully, can capture certain patterns
with high precision.

However, \REs are more capable than that and existing approaches only scratch the surface of what could possibly be achieved with \REs.
When writing a \RE, developers have encoded their knowledge about the problem domain, which can be used to improve data-driven models. For
example, a \RE often indicates informative words (i.e., \textbf{\textit{clue words}}) of certain patterns. If we can exploit such
information, we can then \red{xx}.

This paper presents a novel approach to unlock the potential of \REs in \NLP. Our goal is to enable the use of high-precision \REs (but are
time-consuming to generate) with those of low quality (but are easy to write). We achieve this by employing \NNs, a learning framework that
are widely used in many \NLP tasks~\cite{goldberg2017neural} and are shown to be effective in tolerating noises in the training
data~\cite{xie2016disturblabel}. This allows us to utilize a diverse set of \REs and at the same time to control the inevitable noises of
 low quality \REs. The result is a new way of using \REs, with a generic framework for quality control.

% On the other side, since \NN models rely on distributed representation, they can generalize the \RE pattern so that phrases with similar meaning can also been matched, which may possibly result in better performance than using the \RE alone.

% On the other hand, instead of encoding knowledge into massive labeled data, human tend to express their knowledge in a more compact way, i.e., the rules. The rules accumulated by domain experts makes it an excellent source to compensate the shortage of labeled data.


Specifically, we exploit three ways to combine \REs with a \NN. First, we propose to use the evaluation outcome of \REs as the input
features of a \NN. This shares a similar spirit with the stacking technique~\cite{wolpert1992stacked} for short text
classification~\cite{wang2017combining}. (2) On the \NN module side, since \REs explicitly highlight clue words for a specific tag, we can
use \RE to guide the attention mechanism in \NN. (3) On the \NN output side, we can combine the output of \RE and \NN in
a direct but learnable way, so that the final output benefits from both directly. % \NN and \RE.
%Hu et al.~\shortcite{hu2016harnessing} also explored this aspect by using first-order-logic (\FOL) constraints to
%affect the \NN output in a teacher-student network manner. Different from their general framework, since \RE output is
%usually related to the label that we are trying to predict, we use a more direct way for combination.
% However, since it is a general framework, it sacrifices some directness of the combination for its generality。
% but they are more focused on constraints that should not be broke rather than the positive signals produced by \RE.
%
We experiment our methods in two \red{spoken language understanding (\SLU)} tasks, intent detection
and slot filling, which correspond to two basic \NLP tasks: sentence classification and sequence labeling.
%Further, this is also a task where \RE is heavily used in industry.
%
We explore both the few-shot learning setting where annotation is limited and the setting with full training data,
to see how \RE helps when we have different amounts of annotated data.
%To guide \RE annotation,
%We also examine the impact of \RE complexity to the \NN performance.

Our contributions are as follows. (1) This is the first work to systematically investigate methods of fusing \RE with \NN. (2) The
extensive experiments show that our methods clearly improve the \NN performance in both the few-shot learning and the
full annotation settings. (3) Our analysis provides meaningful guidance to fusion method selection, and \RE annotation as well.

\section{Introduction}

Neural networks (\NN) have been proven to be effective in various natural language processing (\NLP) tasks, including text classification \cite{kim2014convolutional}, question answering \cite{yih2015semantic}, machine translation \cite{bahdanau2014neural}, and etc.

However, the data-driven nature of \NN makes it heavily rely on large amount of labeled data, which is expensive and hard to produce.
% For example, when developing a dialogue system for a new domain where there is no relevant data, we can only come up with several queries based on our experience before we have user logs, which is not enough for \NN to work well.

On the other hand, instead of encoding knowledge into massive labeled data, human tend to express their knowledge in a more compact way, i.e., the rules. The rules accumulated by domain experts makes it an excellent source to compensate the shortage of labeled data.

In practice, regular expressions (\RE) is one of the most commonly used human-generated rules in real word. 
For example, when building a dialogue system, an \RE pattern \texttt{/\textasciicircum flights? from/} can help recognize the sentence in Table \ref{atis_sample} to expressing \emph{flights} intent, which indicates that the user is looking for flight information.
In industry, people often write some high-precision \RE patterns to handle a fraction of cases, and leave the rest for data-driven models. And \RE can also be the dominent method in few-shot learning scenarios, where the training data is limited.

However, the value \RE patterns should not be limited to only handle specific cases in a high-confident way. The knowledge encoded in the pattern can also been used to help data-driven models like \NN. Furthermore, this is also a perfect scenario where the easy-to-produce low-precision patterns can work, since \NN is known to be good at tolerenting noises \cite{srivastava2014dropout, xie2016disturblabel}.

Specifically, there are 3 different ways of fusing \RE with \NN. First, we can use the output of \RE as features to \NN. This shares similar spirit with the stacking technic \cite{wolpert1992stacked} and has also been used by \cite{wangcombining17} to incorporate knowledge base rules into short text classification. Second, since the \RE patterns highlights important phrases for a specific tag, we can use \RE to help guide the attention module in \NN. Third, we can also combine the output of \RE and \NN in a learnable way, so that the final output contains the information from 2 aspects. \cite{hu2016deep} also explored to use rules to modify the \NN output in a posterior regularization manner, but they are more focused on constraints that should not be broke rather than the positive signals produced by \RE.

In this paper, we investigate the Natural Language Understanding (\NLU) problem in dialogue system to explore methods for fusing \RE and \NN. We choose this task because it contains intent detection and slot filling as two subtasks, which corresponds to two of the most important tasks in Natualy Language Processing (\NLP): sentence classification, sequence labeling. 
Furthermore, in industry, this is also a task where \RE patterns are heavily used.
We explore the few-shot learning setting where the data is limited as well as the setting using the full dataset, to see how \RE helps when we have different amount of data. To guide \RE annotation, we also conduct quantitive analysis on the impact of \RE complexity to the performance \NN. 

Our contributions are: (1) To our knowledge, we are the first to systematically investigate methods for combining \RE and \NN. (2) The extensive experiment shows that our method significantly improves the model performance in few-shot learning settings, and also makes improvements when using the full dataset. (3) Our analysis provides meaningful guidance to fusing method selection, and \RE annotation as well.


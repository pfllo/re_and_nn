\section{The Task}
\subsection{Problem Definition}
Intent detection is a typical sentence classification task. Given a set of training examples $\{(x_i, y_i): i=1,...,N\}$, we need to learn a mapping function $f: \mathcal{X} \rightarrow \mathcal{Y}$ that maps an input query sentence $x$ to the corresponding intent $y$.

On the other side, slot labeling is often treated as a sequence labelling problem. Here we also have a set of training examples $\{(\textbf{x}_i, \textbf{y}_i): i=1,...,N\}$, where $\textbf{x}_i=[x_{i1}, ..., x_{in_i}]$ is the input sentence with $n_i$ words, and $\textbf{y}_i=[y_{i1}, ..., y_{in_i}]$ is the tag of each word. We need to learn a mapping function $f: \mathcal{X} \rightarrow \mathcal{Y}$ that maps an input query sentence $\textbf{x}$ to the corresponding tag sequence $\textbf{y}$.

For example, Table \ref{atis_sample} shows a sample sentence in the ATIS dataset. The query wants the information about flights from New York to Miami, so the intent is \emph{flight}, which means the query wants flight-related information, and we also need to identify the \emph{from\_city} and the \emph{to\_city} so that the backend system can return the information that the user need.

\begin{table}
\setlength{\tabcolsep}{0.23em}
\centering
\small{
\begin{tabular}{|c|c|c|c|c|c|}

\hline
flights &from &new &york &to &miami  \\
\hline
O &O &B-fromloc.city &I-fromloc.city &O &B-toloc.city  \\
\hline
\multicolumn{6}{|c|}{flight} \\
\hline
\end{tabular}
}
\caption{ATIS corpus sample with intent and slot annotation. The first line is sentence, the second line is slot filling tags, the third line is intent label.}
\label{atis_sample}
\end{table}

\subsection{Regular Expression Patterns}
Regular expression patterns are commonly used in naturaly language related tasks. As for intent detection, an \RE pattern typically models a text pattern and output a related label for the sentence when the pattern mathes. For example, a pattern 
\textsl{/\textasciicircum flights? from/} that matches sentences which starts with \emph{flights from} for \emph{flight from} can match the sentence in Table \ref{atis_sample}, and then output that the sentence is of intent \emph{flight}.

Similarly, as for slot filling, since \RE has group functionality, we can assign different tags for our intested text groups in the pattern. For example, the pattern \textsl{/from (\_\_CITY) to (\_\_CITY)/} will match the sentence in Tabel \ref{atis_sample}. We can assign the \emph{fromloc.city} tag to the first group (the first part surrounded by brackets), and \emph{toloc.city} to the second group. Here \emph{\_\_CITY} is a list containing all the city names, which can be repalces with a string like \textsl{/new york|miami|LA/} to form actual regular expression. This also makes \RE patterns a natural fit for word list, since an ordinary word list rule can be written as \textsl{/(\_\_CITY)/} in \RE.


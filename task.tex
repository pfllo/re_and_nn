\section{The Task}
\subsection{Problem Definition}
Intent detection is a typical sentence classification task. Given a set of training examples $\{(x_i, y_i): i=1,...,N\}$, we need to learn a mapping function $f: \mathcal{X} \rightarrow \mathcal{Y}$ that maps an input query sentence $x$ to the corresponding intent $y$.

On the other side, slot filling is often treated as a sequence labelling problem. Here we also have a set of training examples $\{(\textbf{x}_i, \textbf{y}_i): i=1,...,N\}$, where $\textbf{x}_i=[x_{i1}, ..., x_{in_i}]$ is the input sentence with $n_i$ words, and $\textbf{y}_i=[y_{i1}, ..., y_{in_i}]$ is the slot label of each word. We need to learn a mapping function $f: \mathcal{X} \rightarrow \mathcal{Y}$ that maps an input query sentence $\textbf{x}$ to the corresponding label sequence $\textbf{y}$.

For example, Table \ref{atis_sample} shows a sample sentence in the ATIS dataset. The intent of the query is \emph{flight}, which means the user wants flight-related information, and we also need to identify the \emph{from\_city} and the \emph{to\_city} slots so that the backend system can return the correct information.

\begin{table}
\setlength{\tabcolsep}{0.23em}
\centering
\small{
\begin{tabular}{|c|l|}

\hline
\textbf{Sentence} &flights \;\;\; from \;\;\;\;\; boston \;\;\;\;\;\;\; to \;\;\;\; miami  \\
\hline
\textbf{Slot Label} &\;\;\; O \;\;\;\;\;\;\;\; O \;\; B-fromloc.city \, O \; B-toloc.city \\
\hline
\textbf{Slot RE} & \multicolumn{1}{|l|}{\quad\quad\quad\;\;/from \quad (\_\_CITY) \;\;\;\;\,\, to \quad (\_\_CITY)/} \\
\hline
\textbf{Intent} &\multicolumn{1}{|c|}{flight} \\
\hline
\textbf{Intent RE} & \multicolumn{1}{|l|}{/\textasciicircum flights? from/} \\
\hline
\end{tabular}
}
\caption{ATIS corpus sample with intent, slots and corresponding \RE patterns.}
\label{atis_sample}
\end{table}


\subsection{Regular Expression Patterns}
Regular expression patterns are commonly used in naturaly language related tasks. As for intent detection, an \RE typically models a text pattern and output a related tag for the sentence when the pattern mathes. For example, a pattern 
\texttt{/\textasciicircum flights?\:from/} can match the sentence in Table \ref{atis_sample}, and then output that the sentence is of intent \emph{flight}. 


Similarly, as for slot filling, given the group functionality of \RE, we can assign different tags for our interested text groups (the text surrounded by brackets) in the pattern. For example, the pattern \texttt{/from\:(\_\_CITY)\:to\:(\_\_CITY)/} will match the sentence in Tabel \ref{atis_sample}, and assign the \emph{fromloc.city} tag to the first group, and \emph{toloc.city} to the second one. Here \emph{\_\_CITY} is a list containing all the city names, which can be repalced with a string like \texttt{/boston|miami|LA/} to form the actual \RE. This also makes \RE patterns a natural fit for word list, since an ordinary word list rule can be written as \texttt{/(\_\_CITY)/} in \RE.

While the outputs are the intent or slot themselves in the examples here, the \RE output can also be a tag related to, but not the same as, the label that we want to predict (see Section \ref{fusion_with_input} and \ref{fusion_with_output}).

When writing \RE patterns, since complicated \texttt{RE}s often requires more efforts to generate, the complexity of the pattern is usually an important trade-off that we need to make. Generally, 2 aspects of an \RE matters the most. First, the pattern complexity increases with the number of groups in the \RE. This kind of complexity will lead to better precision and lower coverage. Second, the pattern complexity also increases with the number of \emph{or}s (the symbol $|$) in a group. Here a group can be considered as a phrase set that expresses the same meaning. Therefore, unless adding ambiguous phrases to the group, this kind of complexity usually results in higher coverage and slightly lower precision. \red{due to limited space, examples for \RE complexity is omitted}  


\section{Background}
\begin{figure}[t!]
  \centering
  % Requires \usepackage{graphicx}
  \includegraphics[width=0.49\textwidth]{figure/motivation.pdf}\\
  \vspace{-2mm}
  \caption{A sentence from the ATIS dataset. \REs can be used to detect the intent and label slots.}
  \label{atis_sample}
  \vspace{-3mm}
\end{figure}

\vspace{-2mm}
\subsection{Problem Definition}
\vspace{-2mm}
%While our approach is generally applicable, to have concrete and measurable objectives,
Our work targets two \SLU tasks: \emph{intent detection} and \emph{slot filling}. The former is a sentence classification task where we
learn a function to map an input sentence of $n$ words, $\textbf{x}=[x_{1}, ..., x_{n}]$, to a corresponding \textbf{\emph{intent label}},
$c$. The latter is a sequence labeling task for which we learn a function to take in an input query sentence of $n$ words,
$\textbf{x}=[x_{1}, ..., x_{n}]$, to produce a corresponding labeling sequence, $\textbf{y}=[y_{1}, ..., y_{n}]$, where  $y_i$ is the
\textbf{\emph{slot label}} of the
corresponding word, $x_i$. % in the input query.


%Intent detection is a typical sentence classification task.
%We need to learn a mapping function that maps an input sentence $\textbf{x}$, where $\textbf{x}=[x_{1}, ..., x_{n}]$ is the input sentence with $n$ words, to the corresponding intent $y$.
%% $f: \mathcal{X} \rightarrow \mathcal{Y}$
%% Given a set of training examples $\{(x_i, y_i): i=1,...,N\}$,
%
%On the other side, slot filling is often treated as a sequence labeling task.
%We need to learn a mapping function that maps an input query sentence $\textbf{x}$ to the corresponding label sequence $\textbf{y}$, where $\textbf{y}=[y_{1}, ..., y_{n}]$ is the slot labels of each word.
% Here we also have a set of training examples $\{(\textbf{x}_i, \textbf{y}_i): i=1,...,N\}$, where $\textbf{x}_i=[x_{i1}, ..., x_{in_i}]$ is the input sentence with $n_i$ words, and $\textbf{y}_i=[y_{i1}, ..., y_{in_i}]$ is the slot label of each word.

%For example, Figure\ref{atis_sample} shows a sample sentence in the ATIS dataset. Its intent is \emph{flight}, indicating the user wants
%flight-related information. We also need to identify the \emph{from\_city} and the \emph{to\_city} slots so that the we can return the
%correct information.

%\cparagraph{Example}
Take the sentence in Fig.~\ref{atis_sample} as an example.
%gives an example sentence from the ATIS (Airline Travel Information Systems) dataset~\cite{hemphill1990atis}.
A successful intent detector would suggest the intent of the sentence as \emph{flight}, i.e., querying
about flight-related information. Slot filler, on the other hand, should identify the slots \emph{fromloc.city} and
\emph{toloc.city} by labeling \underline{\textit{Boston}} and \underline{\textit{Miami}}, respectively,
%the sentence
using the begin-inside-outside (\texttt{BIO}) scheme.




\subsection{The Use of Regular Expressions}
\label{re_desc} \vspace{-2mm}

In this work, a \RE defines a mapping from a text \emph{pattern} to several \textbf{\emph{\REtags}} which are  the same as or related to
the \textbf{\emph{target labels}} (i.e., intent and slot labels). A search function takes in a \RE, applies it to all sentences, and
returns any texts that match the pattern. We assign the \REtag(\texttt{s}) (that are associated with the matching \RE) to either the
matched sentence (for intent detection) or some matched phrases (for slot filling).


% If a \REtag indicates an intent, it would be the same as an intent label.
In this work, the \REtags for intent detection are the same as the intent labels.
For example,  %when applying the intent \RE
in Fig.~\ref{atis_sample},
% to the example sentence,
we get a \REtag of \emph{flight} that is the same as %corresponds to
 the intent label \emph{flight}.


% an \RE of \texttt{/\textasciicircum
% flights?\:from/} can be associated with an intention label \emph{flight}; and when applying this \RE to the sentence given in
% Figure~\ref{atis_sample} for intention detection, we get a \REtag of \emph{flight}.

For slot filling, we use two different sets of \REs. Given the group functionality of \RE, we can assign \REtags to our interested
\textbf{\emph{\RE groups}} (i.e., the expressions defined inside parentheses of a \RE). The translation from \REtags to slot labels depends
on how the corresponding \REs are used. When \REs are used at the network module level (Sec.~\ref{interact_with_module}), the corresponding
\REtags are the same as the target slot labels because our goal is to annotate clue words for certain slot labels. For instance, the slot
\RE in Fig.~\ref{atis_sample} will assign \emph{fromloc.city} to the first \RE group and \emph{toloc.city} to the second one. In this
example, the macro \emph{\_\_CITY} is the list of the target city names, which can be replaced with a string like
\texttt{\small/Boston|Miami|LA|.../}. If \REs are used in the input (Sec.~\ref{fusion_with_input}) and the output layers
(Sec.~\ref{fusion_with_output}) of a \NN, the corresponding \REtag would be different from the target slot label. In this context, the
first and second \RE groups of the slot \RE of Fig.~\ref{atis_sample} would be simply tagged as \emph{city} to capture the commonality of
three target slot labels: \emph{fromloc.city}, \emph{toloc.city}, \emph{stoploc.city}. The purpose of abstracting the \REtag to a
simplified version of multiple target slot labels is to
% maximize the applicability of \REs, so that
show that
\REs can still be useful when their
evaluation outcome does not exactly match our learning objective.
%
%(1)~For the method in Sec.~\ref{interact_with_module}, since we need to annotate clue words for certain slot labels, its \REtags are the
%same as our target slot labels. For example, the slot \RE in Fig.~\ref{atis_sample} will assign \emph{fromloc.city} to the first \RE group
%and \emph{toloc.city} to the second group.
%% \texttt{/(from)\:(\_\_CITY)/} will match the sentence in Figure~\ref{atis_sample}, and assign \emph{fromloc.city} to the second \RE group.
%Here, \emph{\_\_CITY} is a full list of city names, which can be replaced with a string like \texttt{/Boston|Miami|LA|.../}.
%%Note that, an ordinary word list pattern can be written as strings like \texttt{/(\_\_CITY)/} in \RE.
%(2)~For the methods in Sec.~\ref{fusion_with_input} and \ref{fusion_with_output}, the
%\REtag is different from the target slot label.
%For example, the first and second \RE groups in the slot \RE of Fig.~\ref{atis_sample} can be simply tagged as \emph{city},
%% the second \RE group in \texttt{/(from)\:(\_\_CITY)/} will be tagged as \emph{city},
%which is a simplified version of the target slot label, related to three slot labels: \emph{fromloc.city}, \emph{toloc.city}, \emph{stoploc.city}.
%\red{The reason that we use different \REs is to give an example that \REs, with simplified output labels, can also make improvements to
%\NNs.}

% We use another set of \texttt{RE}s for the methods in Sec.~\ref{fusion_with_input} and \ref{fusion_with_output} because using simpler tags can significantly reduce the complexity of the \RE, and therefore making the generation of \RE much easier. For example, we need \texttt{/(from)\:(\_\_CITY)/} to identify \emph{fromloc.city}, but only \texttt{/(\_\_CITY)/} to identify \emph{city}.

%When writing \REs,
Intuitively, complicated \REs can lead to better performance but require more efforts to generate. %, \RE complexity is often an
 Generally, there are mainly two aspects affecting \RE complexity most, the number of \RE groups
and \emph{or} clauses (i.e., expressions separated by the disjunction operator $|$) in a \RE group. Having a larger number of \RE groups
often leads to better precision but lower coverage on pattern matching, while a larger number of \emph{or} clauses usually gives a higher
coverage but slightly lower precision.


% First, \RE complexity increases with the number of \RE groups. This kind of complexity will lead to better precision but lower coverage. Second, \RE complexity also increases with the number of \emph{or}s (the symbol $|$) in a \RE group. Here a group can be considered as a set of phrases that express the same meaning. Therefore, this kind of complexity usually results in higher coverage and slightly lower precision.
% unless adding ambiguous phrases to the group

% Note that, while the outputs are the intent or slot themselves in the examples above, the \RE output can also be a tag related to, but not the same as, the label that we want to predict.
% We allow this variation because using tags different from the target labels can sometimes make it easier to write an \RE. For example, we need \texttt{/from\:(\_\_CITY)/} to identify \emph{fromloc.city}, but only \texttt{/(\_\_CITY)/} to identify \emph{city}. In this paper, the output of intent patterns is the same as the intent label, and the output of slot patterns used by the method in Section \ref{interact_with_module} is the same as the slot label. However, the slot \RE tag for methods in Section \ref{fusion_with_input} and \ref{fusion_with_output} is the entity part of the slot label (e.g., \emph{city} in \emph{fromloc.city}).

\section{The Task}
\subsection{Problem Definition}
Intent detection is a typical sentence classification task.
We need to learn a mapping function that maps an input sentence $\textbf{x}$, where $\textbf{x}=[x_{1}, ..., x_{n}]$ is the input sentence with $n$ words, to the corresponding intent $y$.
% $f: \mathcal{X} \rightarrow \mathcal{Y}$
% Given a set of training examples $\{(x_i, y_i): i=1,...,N\}$, 

On the other side, slot filling is often treated as a sequence labeling task. 
We need to learn a mapping function that maps an input query sentence $\textbf{x}$ to the corresponding label sequence $\textbf{y}$, where $\textbf{y}=[y_{1}, ..., y_{n}]$ is the slot labels of each word.
% Here we also have a set of training examples $\{(\textbf{x}_i, \textbf{y}_i): i=1,...,N\}$, where $\textbf{x}_i=[x_{i1}, ..., x_{in_i}]$ is the input sentence with $n_i$ words, and $\textbf{y}_i=[y_{i1}, ..., y_{in_i}]$ is the slot label of each word. 

For example, Table \ref{atis_sample} shows a sample sentence in the ATIS dataset. Its intent is \emph{flight}, indicating the user wants flight-related information. We also need to identify the \emph{from\_city} and the \emph{to\_city} slots so that the we can return the correct information.

\begin{table}
\setlength{\tabcolsep}{0.23em}
\centering
\small{
\begin{tabular}{|c|l|}

\hline
\textbf{Sentence} &flights \;\;\; from \;\;\;\;\; boston \;\;\;\;\;\;\; to \;\;\;\; miami  \\
\hline
\textbf{Slot Label} &\;\;\; O \;\;\;\;\;\;\;\; O \;\; B-fromloc.city \, O \; B-toloc.city \\
\hline
\textbf{Slot RE} & \multicolumn{1}{|l|}{\quad\quad\quad\;\;/from \quad (\_\_CITY) \;\;\;\;\,\, to \quad (\_\_CITY)/} \\
\hline
\textbf{Intent} &\multicolumn{1}{|c|}{flight} \\
\hline
\textbf{Intent RE} & \multicolumn{1}{|l|}{/\textasciicircum flights? from/} \\
\hline
\end{tabular}
}
\caption{ATIS corpus sample with intent, slots and corresponding \RE patterns.}
\label{atis_sample}
\end{table}


\subsection{Regular Expression Patterns}
\label{re_desc}
We denote the output of \RE as \textbf{\emph{\RE tag}}, and the label that we are trying to predict as \textbf{\emph{intent label}} and \textbf{\emph{slot label}} for intent detection and slot filling respectively, which are generally referred to as \textbf{\emph{target label}}. Note that, while not necessarily the same as the target label, \RE tag is usually related to one or more target labels. For example, the \RE tag \emph{city} is related to both \emph{fromloc.city} and \emph{toloc.city}.

Specifically, in this paper, our \RE tag for intent detection is the same as the intent label. For example, a pattern 
\texttt{/\textasciicircum flights?\:from/} can match the sentence in Table~\ref{atis_sample}, and output intent \emph{flight}. 

However, for slot filling, we use two different set of \texttt{RE}s. 
Given the group functionality of \RE, we can assign tags to our interested groups (the text surrounded by brackets). 
(1)~For the method in Sec.~\ref{interact_with_module}, since we need to annotate clue words for certain slot labels, its \RE tag is the same as the slot labels. 
For example, \texttt{/(from)\:(\_\_CITY)/} will match the sentence in Tabel~\ref{atis_sample}, and assign \emph{fromloc.city} to the second group. 
Here \emph{\_\_CITY} is a list of all the city names, which can be repalced with a string like \texttt{/boston|miami|LA/}. 
Note that, an ordinary word list pattern can be written as strings like \texttt{/(\_\_CITY)/} in \RE.
(2)~For the methods in Sec.~\ref{fusion_with_input} and \ref{fusion_with_output}, the \RE tag is different from the slot label. For example, the second group in \texttt{/(from)\:(\_\_CITY)/} will be tagged as \emph{city}, which is a simplified version of the slot label. The reason that we use different \texttt{RE}s here is to give an example that \texttt{RE}s with output different from the target labels can also make improvements to \NN.

% We use another set of \texttt{RE}s for the methods in Sec.~\ref{fusion_with_input} and \ref{fusion_with_output} because using simpler tags can significantly reduce the complexity of the \RE, and therefore making the generation of \RE much easier. For example, we need \texttt{/(from)\:(\_\_CITY)/} to identify \emph{fromloc.city}, but only \texttt{/(\_\_CITY)/} to identify \emph{city}.

When writing \RE patterns, since complicated \texttt{RE}s often leads to better performance but require more efforts to generate, pattern complexity is usually an important trade-off that we need to make. Generally, two aspects of \RE influence the complexity the most. First, pattern complexity increases with the number of groups in the \RE. This kind of complexity will lead to better precision and lower coverage. Second, pattern complexity also increases with the number of \emph{or}s (the symbol $|$) in a group. Here a group can be considered as a set of phrases that express the same meaning. Therefore, this kind of complexity usually results in higher coverage and slightly lower precision. 
% unless adding ambiguous phrases to the group

% Note that, while the outputs are the intent or slot themselves in the examples above, the \RE output can also be a tag related to, but not the same as, the label that we want to predict.
% We allow this variation because using tags different from the target labels can sometimes make it easier to write an \RE. For example, we need \texttt{/from\:(\_\_CITY)/} to identify \emph{fromloc.city}, but only \texttt{/(\_\_CITY)/} to identify \emph{city}. In this paper, the output of intent patterns is the same as the intent label, and the output of slot patterns used by the method in Section \ref{interact_with_module} is the same as the slot label. However, the slot \RE tag for methods in Section \ref{fusion_with_input} and \ref{fusion_with_output} is the entity part of the slot label (e.g., \emph{city} in \emph{fromloc.city}).



